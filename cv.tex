% resume.tex
% vim:set ft=tex spell:

\documentclass[10pt,letterpaper]{article}
\usepackage[letterpaper,margin=0.75in]{geometry}
\usepackage[utf8]{inputenc}
\usepackage{mdwlist}
\usepackage[T1]{fontenc}
\usepackage{textcomp}
\usepackage{tgpagella}
\usepackage{latexsym}
\usepackage{amssymb}
\usepackage{hyperref}
\usepackage[usenames, dvipsnames]{color}
\pagestyle{empty}
\setlength{\tabcolsep}{0em}

% indentsection style, used for sections that aren't already in lists
% that need indentation to the level of all text in the document
\newenvironment{indentsection}[1]%
{\begin{list}{}%
	{\setlength{\leftmargin}{#1}}%
	\item[]%
}
{\end{list}}

% opposite of above; bump a section back toward the left margin
\newenvironment{unindentsection}[1]%
{\begin{list}{}%
	{\setlength{\leftmargin}{-0.5#1}}%
	\item[]%
}
{\end{list}}

% format two pieces of text, one left aligned and one right aligned
\newcommand{\headerrow}[2]
{\begin{tabular*}{\linewidth}{l@{\extracolsep{\fill}}r}
	#1 &
	#2 \\
\end{tabular*}}

% make "C++" look pretty when used in text by touching up the plus signs
\newcommand{\CPP}
{C\nolinebreak[4]\hspace{-.05em}\raisebox{.22ex}{\footnotesize\bf ++}}

% and the actual content starts here
\begin{document}

\begin{center}
{\LARGE \color{MidnightBlue} \textbf{Kumar Shridhar}}

{\color{BrickRed}\url{ kumar-shridhar.github.io}\color{BrickRed}} \textbullet
\ \ \texttt{shridhar.stark@gmail.com} \textbullet
\ \ +49 176 77659867
\\
Kurt-Schumacher Strasse 16, Kaiserslautern, Germany  67663
\end{center}


\hrule
\vspace{-0.4em}
\subsection*{\color{MidnightBlue}Publications}

\begin{enumerate}
	\parskip=0.1em
	
	\item Joonho Lee*, \textbf{Kumar Shridhar*} \footnote{* Equal Contribution}, Hideaki Hayashi, Brian Kenji Iwana, Seokjun Kang, Seiichi Uchida (2019). \href{https://arxiv.org/abs/1905.10761}{ProbAct: A Probabilistic Activation Function for Deep Neural Networks}\footnote{\url{https://arxiv.org/abs/1806.05978}}. \emph{Submitted to 33rd Conference on Neural Information Processing Systems (NeurIPS 2019).}
	
	\item\textbf{Kumar Shridhar}, Felix Laumann, Marcus Liwicki (2019). \href{https://arxiv.org/abs/1806.05978}{Uncertainty Estimations by Softplus normalization in Bayesian Convolutional Neural Networks with Variational Inference}\footnote{\url{https://arxiv.org/abs/1806.05978}}. \emph{Submitted to 33rd Conference on Neural Information Processing Systems (NeurIPS 2019).}
	
	\item\textbf{Kumar Shridhar}, Amit Sahu, Ayushman Dash, Pedro Alonso, Gustav Pihlgren, Vinay Pondeknath,  Gyorgy Kovacs, Fotini Simistira, Marcus Liwicki (2018). \href{https://arxiv.org/abs/1810.07150}{Subword Semantic Hashing for Intent Classification on Small Datasets}\footnote{\url{https://arxiv.org/abs/1810.07150}}. \emph{In Proceedings of IJCNN 2019, Budapest, Hungary.}
	
	\item\textbf{Kumar Shridhar},  Felix Laumann, Marcus Liwicki (2019). \href{https://arxiv.org/abs/1901.02731}{A Comprehensive guide to Bayesian Convolutional Neural Network with Variational Inference}\footnote{\url{https://arxiv.org/abs/1806.05978}}. \emph{ArXiv Preprint, arXiv:1901.02731.}
	
	%add abstract
\end{enumerate}


\hrule
\vspace{-0.4em}
\subsection*{\color{MidnightBlue}Experience}

\begin{itemize}
	\parskip=0.1em
	
	\item
	\headerrow
		{\textbf {\href{https://www.botsupply.ai/}{\color{BrickRed}BOTSUPPLY}}}
		{\textbf{Copenhagen, Denmark}}
	\\
	\headerrow
		{\emph{\color{OliveGreen}Chief AI Scientist}}
		{\emph{12/2016 -- 12/2017}}
	\begin{itemize*}
		\item Developed a Natural Language Processing Framework \footnote{\url{https://www.botsupply.ai/natural-language-processing}} from scratch in 40+ languages   that powers all the customers chatbots at BotSupply\footnote{\url{https://www.botsupply.ai/}}.
		\item Developed and trained models for Intent classification, Entity recognition, Sentiment Analysis, Language Translation, POS tagging that are on par with the state-of-the-art models.  
		\item Designed architectures for handling imbalanced datasets, improving performance with continuous learning over feedback and automated selection of the  best threshold. 
		\item Gathered data and feedbacks from real users, crowd-sourced annotations, worked with linguists and designers  to improve the whole conversational flow in chatbots.
		\item My current work focuses on learning representations from non-labeled datasets in an unsupervised manner that generalizes well to any tasks when fine-tuned upon.
	\end{itemize*}

	\item
	\headerrow
		{\textbf{\href{https://www.insiders-technologies.de/home.html}{\color{BrickRed}INSIDERS TECHNOLOGIES}}}
		{\textbf{Kaiserslautern, Germany}}
	\\
	\headerrow
		{\emph{\color{OliveGreen}Research Assistant}}
		{\emph{01/2018 -- 09/2018}}
	\begin{itemize*}	
		\item Worked with the Ovation Machine Learning Team that handles big data, reads and understands the content and interacts with end users through Conversational Intelligent Bots.
		\item My work involved understanding the client problem, design suitable solutions and architectures for different clients and researching to improve the model performance on scarce datasets. 
		\item Contributed to Ovation Framework for Conversational Intelligence \footnote{\url{https://github.com/mindgarage/Ovation}} in collaboration with Mindgarage and participated in Ovation Summer Academy 2017.
	\end{itemize*}

	\item
	\headerrow
		{\textbf{\href{http://mindgarage.ai/}{\color{BrickRed}MINDGARAGE}}}
		{\textbf{Kaiserslautern, Germany}}
	\\
	\headerrow
		{\emph{\color{OliveGreen}Researcher}}
		{\emph{2016 -- Present}}
	\begin{itemize*}
		\item Collaborating and researching on various deep learning algorithms like Bayesian Neural Networks, Memory and Attention models and Object detection.
		\item Part of organizational activities at Mindgarage: Assisting students' projects and masters thesis, organizing hackathons and research colloquiums, and in website and github maintenance.
		\newpage
		\item Assisted in organizing the coursework and assignments for {\emph{Very Deep Learning}} \footnote{\url{https://www.informatik.uni-kl.de/en/studium/lehrveranstaltungen/modulhb/#mod-89-7157}} lectures at TU Kaiserslautern under Prof. Marcus Liwicki. 
	\end{itemize*}
	\item
	\headerrow
		{\textbf{\href{https://www.whizleads.com/}{\color{BrickRed}WHIZLEADS}}}
		{\textbf{Sydney, Australia}}
	\\
	\headerrow
		{\emph{\color{OliveGreen}Machine Learning Engineer}}
		{\emph{10/2016 -- 12/2016}}
	\begin{itemize*}
		\item Worked in development of a suite of sales solutions: insights about clients, lead generation, task and invoice management, and social media integration.
		\item Used machine learning algorithms to generate up to date and meaningful insights about clients' personalities, mood, consumer needs, language style and values using social media data.
		\item Deployed the solution to predict in real time (with every update on social media).
	\end{itemize*}

\end{itemize}


\hrule
\vspace{-0.4em}
\subsection*{\color{MidnightBlue}Education}

\begin{itemize}
	\parskip=0.1em
	
	\item 
	\headerrow
		{\textbf{\color{BrickRed}University of Kaiserslautern}}
		{\textbf{Kaiserslautern, Germany}}
	%add courses taken
	\\
	\headerrow
		{\emph{\color{OliveGreen}Department of Computer Science, Masters}}
		{\emph{04/2016 -- Present}}
	\begin{itemize*}
	    \textbullet Major in Computer Science  \hspace{30mm}
	    \textbullet Minor in Psychology
		\item My curriculum \footnote{\url{ https://www.cs.uni-kl.de/en/studium/studiengaenge/bm-inf/sp.ma/}}includes these subjects but not limited to: Machine Learning I, Very Deep Learning, Applications of Artificial Intelligence, Social Web Mining, 2D Computer Vision, Collaborative Intelligence, Embedded Intelligence, Document and Content Analysis, Linguistics and Language Processing and Neural Basis of Brain. 

        \item The coursework gave a deeper understanding in the areas of artificial intelligence, machine learning, pattern recognition, and computer vision.
	\end{itemize*}
	
	\item 
	\headerrow
		{\textbf{\color{BrickRed}Fast.ai}}
		{\textbf{International Fellowship Student}}
	\\
	%reduce the point to one
	\headerrow
		{\emph{\color{OliveGreen}Deep Learning}}
		{\emph{2017 -- 2017}}
	\begin{itemize*}
		\item I learned to apply cutting-edge Deep Learning methods for Natural Language Processing, Computer Vision and Recommendation Systems to achieve state of the art results more efficiently. 

        \item The course helped a lot in understanding and experimenting with more deeply connected architectures with less computational power and to understand the underline thought behind to further improve it. 
        The primary library used was PyTorch which provides great flexibility in experimenting with new things.
	\end{itemize*}
	
	\item 
	\headerrow
		{\textbf{\color{BrickRed}Lule\aa \ University of Technology, Sweden}}
		{\textbf{Lule\aa \ , Sweden}}
	%add courses taken
	\\
	\headerrow
		{\emph{\color{OliveGreen}Student Researcher}}
		{\emph{02/2019 -- 03/2019}}
	\begin{itemize*}
	
		\item Worked with \href{https://www.ltu.se/staff/m/marliw-1.177225?l=en} {Prof. Marcus Liwicki} and \href{https://www.ltu.se/research/subjects/Maskininlarning?l=en}{EISLAB Machine Learning}, Luleå in Document Analysis and NLP domain. We proposed \href{https://chatbotslife.com/know-your-intent-sota-results-in-intent-classification-8e1ca47f364c}{Subword Semantic Hashing technique for Intent classification}, and achieved state-of-the-art results on three standard text datasets (Chatbot, Web-Applications and Ubuntu datasets). 

        \item We also evaluated the use of rectangular convolution for text based documents for rectangular images against their square counterparts. State-of-the-art results were achieved on Tobacco datasets.
	\end{itemize*}
	\item 
	\headerrow
		{\textbf{\color{BrickRed}Kyushu University}}
		{\textbf{Fukuoka, Japan}}
	%add courses taken
	\\
	\headerrow
		{\emph{\color{OliveGreen}Visiting Researcher}}
		{\emph{04/2019 -- 05/2019}}
	\begin{itemize*}
		\item Worked with \href{http://human.ait.kyushu-u.ac.jp/~uchida/index-e.html} {Prof. Seiichi Uchida} in \href{http://human.ait.kyushu-u.ac.jp/index.html}{Human Interface Lab}, Fukuoka Japan on probabilistic activation function (under review in NeurIPS 2019)  

        \item Collaborating further with two other members at the lab to extend the work in the domain of Generative Adversarial Network.
	\end{itemize*}
	
	
\end{itemize}

\hrule
\vspace{-0.4em}
\subsection*{\color{MidnightBlue}Projects}

\begin{itemize}
	\parskip=0.1em
	
	\item 
	\headerrow
		{\textbf{\color{BrickRed}Bayesian Convolutional Neural Network using Variational Inference}}
			{\textbf{}}
	\\
	%put at one place
	\headerrow
		{\emph{\color{OliveGreen}Bayesian Convolutional Neural Network based on Bayes by Backprop in PyTorch}}
		{\emph{}}
	\begin{itemize*}
		\item A proposed Bayes by Backprop CNN framework with various network architectures that performs comparable to convolutional neural networks with point-estimates weights. This work symbolizes the extension of the group of Bayesian neural networks to CNN. Uncertainties estimates were computed and the concept was applied to Image Recognition, Image Super-Resolution and Generative Adversarial Networks.
		\item {\url{https://github.com/kumar-shridhar/PyTorch-BayesianCNN}}

	\end{itemize*}
	
	\item 
	\headerrow
		{\textbf{\color{BrickRed}Know your Intent: Semantic Hashing as Featurizer}}
			{\textbf{}}
	\\
	\headerrow
		{\emph{\color{OliveGreen}Semantic Hashing for Robust Text Classification with small data-sets}}
		{\emph{}}
	\begin{itemize*}
		\item Using Semantic Hashing technique inspired from Deep Semantic Similarity model to overcome problems of out-of-vocabulary terms and spelling mistakes in small datasets for Intent Classification task. This work depends on using hash values as featurizers. State-of-the-art results were achieved on three datasets (AskUbuntu, WebApplication and Chatbot)
        \item {\url{https://github.com/kumar-shridhar/Know-Your-Intent}}
	\end{itemize*}
	
		\item 
	\headerrow
		{\textbf{\color{BrickRed}Text Super Resolution}}
			{\textbf{}}
	\\
	\headerrow
		{\emph{\color{OliveGreen}Superresolution using an efficient sub-pixel convolutional neural network in PyTorch}}
		{\emph{}}
	\begin{itemize*}
		\item Super resolution of text documents using efficient sub-pixel convolutional neural network to improve the performance of OCR. This work was done as a part of Hackathon organized at Mindgarage.
		\item {\url{https://github.com/kumar-shridhar/super_resolution_PyTorch}}

	\end{itemize*}
	
		\item 
	\headerrow
		{\textbf{\color{BrickRed}Political Affiliation Prediction - Twitter}}
			{\textbf{}}
	\\
	\headerrow
		{\emph{\color{OliveGreen}Predicting Political Affiliation of users based on Twitter Data (Tweets) in TensorFlow}}
		{\emph{}}
	\begin{itemize*}
		\item Users' affiliation towards a German political party was predicted using word embeddings as featurizers and a CNN as a classifier. Results were further analyzed and a short paper and poster were presented. This work was a part of my academic curriculum.
        \item {\url{https://github.com/kumar-shridhar/Twitter_Political_Party_Prediction}}
	\end{itemize*}
	
\end{itemize}

\hrule
\vspace{-0.4em}
\subsection*{\color{MidnightBlue}Certificates and awards}

\begin{itemize}
	\parskip=0.1em
	
	\item 
	\headerrow
		{{\textbf{\href{https://www.kaggle.com/shridhar743}{Kaggle}}} Top 1\% -- Plant Seedling Identification}
		{\emph{11/2017 -- Present}}
	\item 
	\headerrow
		{{\textbf{\href{https://medium.com/@shridhar743}{Medium}}} Top Writer -- Artificial Intelligence}
		{\emph{07/2017 -- 09/2017}}
	\item 
	\headerrow
		{Member of Botsupply IBM Award Winner 2017 Team}
		{\emph{11/2017}}	

\end{itemize}

\hrule
\vspace{-0.4em}
\subsection*{\color{MidnightBlue}Languages and Technologies}

\begin{indentsection}{\parindent}
\hyphenpenalty=1000
\begin{description*}
	\item[\color{BrickRed}Programming Languages:]
	Python, C, C++
	\item[\color{BrickRed}Technologies:]
	PyTorch, Keras, TensorFlow, SciPy, NumPy, scikit-learn, NLTK, RASA, SpaCy, CoreNLP, UNIX, Docker, Git, \LaTeX
	\item[\color{BrickRed}Natural Languages:]
	Native in English and Hindi, intermediate in German	
	\item[\color{BrickRed}Open Source Contributions:]
	Facebook Duckling \footnote{\url{ https://github.com/kumar-shridhar/duckling}}, ContinualAI \footnote{\url{https://github.com/ContinualAI}}
\end{description*}
\end{indentsection}

\hrule
\vspace{-0.4em}
\subsection*{\color{MidnightBlue} Research Collaborations}

\begin{itemize}
	\parskip=0.1em
	
	\item 
	\headerrow
		{\textbf{\href{http://www.mobile-industrial-robots.com/en/}{\color{BrickRed}Mobile Industrial Robots}}}
			{\textbf{}}
	\\
	\headerrow
		{\emph{\color{OliveGreen}Improvement of Object detection and localization systems in Mobile Industrial Robots}}
		{\emph{}}
	\begin{itemize*}
		\item Worked in the area of real-time Object detection in Mobile Industrial Robots using Nvidia Jetson devices and Raspberry Pi v2 cameras. Further, experimentation with Intel Movidius devices to reduce overall cost without a reduction in overall performance and accuracy. 

	\end{itemize*}
	
    	\item 
	\headerrow
		{\textbf{\href{https://www.jatana.ai/}{\color{BrickRed}Jatana AI}}}
			{\textbf{}}
	\\
	\headerrow
		{\emph{\color{OliveGreen}Research on learning from feedbacks in a coversational intelligent system}}
		{\emph{}}
	\begin{itemize*}
		\item Working together with researchers at Jatana to make the model learn from customer feedbacks automatically in order to improve the confidence of the low confidence queries replies.  

	\end{itemize*}
	
\end{itemize}

\hrule
\vspace{-0.4em}
\subsection*{\color{MidnightBlue}Organizational activities}

\begin{itemize}
	\parskip=0.1em
		\item 
	\headerrow
		{\textbf{\color{BrickRed}Copenhagen Chatbots and AI Meetup: Organizer}}
		{\emph{07/2017 -- Present}}
	\begin{itemize*}
		\item Organized several chatbots and AI meetups \footnote{\url{https://www.facebook.com/groups/141962696210850/}} to connect researchers, and industry professionals.
		
	\end{itemize*}
	
	\item 
	\headerrow
		{\textbf{\color{BrickRed}MindStorm Open Research Forum: Organizer}}
		{\emph{01/2018 -- 05/2018}}
	\begin{itemize*}
		\item Organized open research forums at Mindgarage \footnote{\url{https://www.facebook.com/events/346701135850291/}} to connect students and researchers to discuss and solve open AI problems.
		
	\end{itemize*}
	
		\item 
	\headerrow
		{\textbf{\color{BrickRed}Hackathons: Organizer}}
		{\emph{10/2017 -- 04/2018}}
	\begin{itemize*}
		\item Organized open end hackathons at Mindgarage \footnote{\url{https://www.facebook.com/events/602280003465979/}} with the aim to find the best solution for a machine learning challenge in one night. 
		
	\end{itemize*}
	

\end{itemize}

\hrule
\vspace{-0.4em}
\subsection*{\color{MidnightBlue}Talks and Presentations}

\begin{itemize}
	\parskip=0.1em
	
	\item {\color{BrickRed} Copenhagen Chatbots and AI Meetup}, June 2017 - Present : Best practices, ongoing research in NLP and combination of chatbots with design process to achieve best results.
	
	\item {\color{BrickRed} Lule\aa\ Technical University}, Lule\aa\ Sweden, August 2018: Know your Intent: Intent classification using Semantic Hashing
	
	\item {\color{BrickRed} iMuSciCA}, Athens Greece, May 2018: Generative Adversarial Networks for Semantic Segmentation 
	\item {\color{BrickRed} Technical University Kaiserslautern}, March 2018: Empirical Evaluation of DenseNet 
		
	\item {\color{BrickRed} Ovation Summer Academy}, September 2017: NER using synthetic datasets 
	
	\item {\color{BrickRed} TechFestival}, Copenhagen Denmark, September 2017: Generative AI 
	
\end{itemize}


\end{document}

