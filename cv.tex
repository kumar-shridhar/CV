% resume.tex
% vim:set ft=tex spell:

\documentclass[10pt,letterpaper]{article}
\usepackage[letterpaper,margin=0.75in]{geometry}
\usepackage[utf8]{inputenc}
\usepackage{mdwlist}
\usepackage[T1]{fontenc}
\usepackage{textcomp}
\usepackage{tgpagella}
\usepackage{latexsym}
\usepackage{amssymb}
\usepackage{hyperref}
\pagestyle{empty}
\setlength{\tabcolsep}{0em}

% indentsection style, used for sections that aren't already in lists
% that need indentation to the level of all text in the document
\newenvironment{indentsection}[1]%
{\begin{list}{}%
	{\setlength{\leftmargin}{#1}}%
	\item[]%
}
{\end{list}}

% opposite of above; bump a section back toward the left margin
\newenvironment{unindentsection}[1]%
{\begin{list}{}%
	{\setlength{\leftmargin}{-0.5#1}}%
	\item[]%
}
{\end{list}}

% format two pieces of text, one left aligned and one right aligned
\newcommand{\headerrow}[2]
{\begin{tabular*}{\linewidth}{l@{\extracolsep{\fill}}r}
	#1 &
	#2 \\
\end{tabular*}}

% make "C++" look pretty when used in text by touching up the plus signs
\newcommand{\CPP}
{C\nolinebreak[4]\hspace{-.05em}\raisebox{.22ex}{\footnotesize\bf ++}}

% and the actual content starts here
\begin{document}

\begin{center}
{\LARGE \textbf{Kumar Shridhar}}

\url{kumar-shridhar.github.io} \textbullet
\ \ \texttt{shridhar.stark@gmail.com} \textbullet
\ \ +49 176 77659867
\\
Kurt-Schumacher Strasse 16, Kaiserslautern, Germany  67663
\end{center}


\hrule
\vspace{-0.4em}
\subsection*{Experience}

\begin{itemize}
	\parskip=0.1em
	
	\item
	\headerrow
		{\textbf{\href{https://www.botsupply.ai/}{BOTSUPPLY}}}
		{\textbf{Copenhagen, Denmark}}
	\\
	\headerrow
		{\emph{Chief AI Scientist}}
		{\emph{12/2016 -- Present}}
	\begin{itemize*}
		\item Developed a Natural Language Processing Framework \footnote{\url{https://www.botsupply.ai/natural-language-processing}} from scratch in 40+ languages   that powers all the customers chatbots at BotSupply\footnote{\url{https://www.botsupply.ai/}}.
		\item Created and trained models for Intent classification, Entity recognition, Sentiment Analysis, Language Translation, POS tagging that are in par with state-of-the-art models.  
		\item Designed architectures for handling imbalanced datasets, for improving performance with continuous learning over feedback and for automated selection of the  best threshold. 
		\item Gathered data and feedbacks from real users, crowd-sourced annotations, worked with linguists and designers  to improve the whole conversational flow in chatbots.
		\item My current work focuses on learning representations from unsupervised datasets that generalizes well to any tasks when fine tuned upon.
	\end{itemize*}

	\item
	\headerrow
		{\textbf{\href{https://www.insiders-technologies.de/home.html}{INSIDERS TECHNOLOGIES}}}
		{\textbf{Kaiserslautern, Germany}}
	\\
	\headerrow
		{\emph{Research Assistant}}
		{\emph{01/2018 -- 09/2018}}
	\begin{itemize*}	
		\item Worked in the Ovation Machine Learning Team of Insiders that handles huge amounts of data, reads and understands their content, handles queries or interacts with end users through Conversational Intelligent Bots.
		\item My work involved finding the most suitable and accurate model based on the client dataset and to improve the model performance on scarce datasets. 
		\item Contributed to Ovation Framework for Conversational Intelligence \footnote{\url{https://github.com/mindgarage/Ovation}} in collaboration with Mindgarage and participated in Ovation Summer Academy 2017.
	\end{itemize*}

	\item
	\headerrow
		{\textbf{\href{http://mindgarage.ai/}{MINDGARAGE}}}
		{\textbf{Kaiserslautern, Germany}}
	\\
	\headerrow
		{\emph{Researcher}}
		{\emph{2016 -- Present}}
	\begin{itemize*}
		\item Collaborating and researching on various deep learning algorithms like Bayesian Neural Networks, Memory and Attention models and Object detection.
		\item Assisting in various organizational activities at Mindgarage including, but not limited to: Assisting students' projects and thesis, organizing hackathons and research colloquiums, website and page maintenance, and so on.
		\item Assisted in organizing the coursework and assignments for \emph{Very Deep Learning} lectures at TU Kaiserslautern under Prof. Marcus Liwicki. 
	\end{itemize*}
	
	\item
	\headerrow
		{\textbf{\href{https://www.whizleads.com/}{WHIZLEADS}}}
		{\textbf{Sydney, Australia}}
	\\
	\headerrow
		{\emph{Machine Learning Engineer}}
		{\emph{10/2016 -- 12/2016}}
	\begin{itemize*}
		\item Worked in development of a suite of sales solutions including lead generation, task and invoice management, social media integration and most importantly insights about clients.
		\item Used machine learning algorithms to generate up to date and meaningful insights about clients' personalities, mood, consumer needs, language style and values using social media data.
		\item Worked in making the whole solution real time to be displayed in the application on every update on social media platforms.  
		
	\end{itemize*}

\end{itemize}

\hrule
\vspace{-0.4em}
\subsection*{Publications}

\begin{enumerate}
	\parskip=0.1em
	
	
	\item Felix Laumann, \textbf{Kumar Shridhar}, Adrian Llopart Maurin (2018). \href{https://arxiv.org/abs/1806.05978v2}{Bayesian Convolutional Neural Networks}. arXiv preprint arXiv:1806.05978v2.
	

\end{enumerate}

\hrule
\vspace{-0.4em}
\subsection*{Education}

\begin{itemize}
	\parskip=0.1em
	
	\item 
	\headerrow
		{\textbf{University of Kaiserslautern}}
		{\textbf{Kaiserslautern, Germany}}
	\\
	\headerrow
		{\emph{Department of Computer Science, Masters}}
		{\emph{04/2016 -- Present}}
	\begin{itemize*}
		\item My coursework deals with making computers behave "intelligently": computers that understand images, speech, and texts, software that reasons, plans, and makes autonomous decisions; systems that interpret sensor data and user behavior and communicate and collaborate with users. 

        \item I got a deeper understanding in the areas of artificial intelligence, machine learning, pattern recognition, and computer vision by learning the core concepts and putting it to use in real life.
	\end{itemize*}
	
	\item 
	\headerrow
		{\textbf{Fast.ai}}
		{\textbf{International Fellowship Student}}
	\\
	\headerrow
		{\emph{Deep Learning}}
		{\emph{2017 -- 2017}}
	\begin{itemize*}
		\item I learned to apply cutting edge Deep Learning methods for Natural Language Processing, Computer Vision and Recommendation Systems to achieve state of the art results more efficiently. 

        \item The course helped a lot in understanding and experimenting with more deeply connected architectures with less computational power and to understand the underline thought behind the ideas and to further improve it. 
        The primary library used was PyTorch which provides great flexibility in experimenting with new things.
	\end{itemize*}
	
	
\end{itemize}

\hrule
\vspace{-0.4em}
\subsection*{Projects}

\begin{itemize}
	\parskip=0.1em
	
	\item 
	\headerrow
		{\textbf{Bayesian ConvNet}}
			{\textbf{}}
	\\
	\headerrow
		{\emph{Bayesian Convolutional Neural Network based on Bayes by Backprop in PyTorch}}
		{\emph{}}
	\begin{itemize*}
		\item A proposed Bayes by Backprop CNN framework with various network architectures that performs comparable to convolutional neural networks with point-estimates weights. This work symbolizes the extension of the group of Bayesian neural networks to CNN.
		\item {\url{https://github.com/kumar-shridhar/PyTorch-BayesianCNN}}

	\end{itemize*}
	
		\item 
	\headerrow
		{\textbf{Text Super Resolution}}
			{\textbf{}}
	\\
	\headerrow
		{\emph{Superresolution using an efficient sub-pixel convolutional neural network in PyTorch}}
		{\emph{}}
	\begin{itemize*}
		\item Super resolution of text documents using efficient sub-pixel convolutional neural network to improve the performance of OCR. This work was done as a part of Hackathon organized at Mindgarage.
		\item {\url{https://github.com/kumar-shridhar/super_resolution_PyTorch}}

	\end{itemize*}
	
			\item 
	\headerrow
		{\textbf{Predicting Political Affiliation - Twitter}}
			{\textbf{}}
	\\
	\headerrow
		{\emph{Predicting Political Affiliation of users based on Twitter Data (Tweets) in TensorFlow}}
		{\emph{}}
	\begin{itemize*}
		\item Users' affiliation towards a German political party was predicted using word embeddings as featurizers and a CNN as a classifier. Results were further analyzed and a short paper and poster were presented. This work was a part of my academic curriculum.
        \item {\url{https://github.com/kumar-shridhar/Twitter_Political_Party_Prediction}}
	\end{itemize*}
	
	
\end{itemize}

\hrule
\vspace{-0.4em}
\subsection*{Certificates and awards}

\begin{itemize}
	\parskip=0.1em
	
	\item 
	\headerrow
		{Kaggle Top 1\% -- Plant Seedling Identification}
		{\emph{11/2017 -- Present}}
	\item 
	\headerrow
		{Medium Top Writer -- Artificial Intelligence}
		{\emph{07/2017 -- 09/2017}}
	\item 
	\headerrow
		{Member of Botsupply IBM Award Winner 2017 Team}
		{\emph{11/2017}}	

\end{itemize}

\hrule
\vspace{-0.4em}
\subsection*{Languages and Technologies}

\begin{indentsection}{\parindent}
\hyphenpenalty=1000
\begin{description*}
	\item[Programming Languages:]
	Python, C, C++
	\item[Technologies:]
	PyTorch, Keras, TensorFlow, SciPy, NumPy, scikit-learn, NLTK, RASA, SpaCy, CoreNLP, UNIX, Docker, Git, \LaTeX
	\item[Natural Languages:]
	Native in English and Hindi, intermediate in German	
	\item[Open Source Contributions:]
	Facebook Duckling
\end{description*}
\end{indentsection}

\hrule
\vspace{-0.4em}
\subsection*{Collaborations}

\begin{itemize}
	\parskip=0.1em
	
	\item 
	\headerrow
		{\textbf{\href{http://www.mobile-industrial-robots.com/en/}{Mobile Industrial Robots}}}
			{\textbf{}}
	\\
	\headerrow
		{\emph{Improving Object detection and localization systems in Mobile Industrial Robots}}
		{\emph{}}
	\begin{itemize*}
		\item Worked in the area of real time Object detection in Mobile Industrial Robots using Nvidia Jetson devices and Raspberry Pi v2 cameras. Further, experimentation with Intel Movidius devices to reduce overall cost without reduction in overall performance and accuracy. 

	\end{itemize*}
	
    	\item 
	\headerrow
		{\textbf{\href{https://www.jatana.ai/}{Jatana AI}}}
			{\textbf{}}
	\\
	\headerrow
		{\emph{Research on learning from feedbacks}}
		{\emph{}}
	\begin{itemize*}
		\item Working together with researchers at Jatana to make the model learn from customer feedbacks automatically in order to improve the confidence of the low confidence queries replies.  

	\end{itemize*}
	
\end{itemize}

\hrule
\vspace{-0.4em}
\subsection*{Organizational activities}

\begin{itemize}
	\parskip=0.1em
		\item 
	\headerrow
		{\textbf{Copenhagen Chatbots and AI Meetup: Organizer}}
		{\emph{07/2017 -- Present}}
	\begin{itemize*}
		\item Organized several chatbots and AI meetups \footnote{\url{https://www.facebook.com/groups/141962696210850/}} to connect researchers, and industry professionals.
		
	\end{itemize*}
	
	\item 
	\headerrow
		{\textbf{MindStorm Open Research Forum: Organizer}}
		{\emph{01/2018 -- 05/2018}}
	\begin{itemize*}
		\item Organized open research forums at Mindgarage \footnote{\url{https://www.facebook.com/events/346701135850291/}} to connect students and researchers to discuss and solve open AI problems.
		
	\end{itemize*}
	
		\item 
	\headerrow
		{\textbf{Hackathons: Organizer}}
		{\emph{10/2017 -- 04/2018}}
	\begin{itemize*}
		\item Organized open end hackathons at Mindgarage \footnote{\url{https://www.facebook.com/events/602280003465979/}} with the aim to find best solution for a machine learning challenge in one night. 
		
	\end{itemize*}
	

\end{itemize}

\hrule
\vspace{-0.4em}
\subsection*{Talks and Presentations}

\begin{itemize}
	\parskip=0.1em
	
	\item Copenhagen Chatbots and AI Meetup, June 2017 - Present : Talks about best practices and ongoing researchs in Chatbots and NLP and how chatbots needs to be combined with design process to achieve best results.
	
	\item iMuSciCA, Athens Greece, May 2018: Generative Adversarial Networks for Semantic Segmentation 
	
	\item Technical University Kaiserslautern, March 2018: Empirical Evaluation of DenseNet 
		
	\item Ovation Summer Academy, September 2017: NER using synthetic datasets 
	
	\item TechFestival, Copenhagen Denmark, September 2017: Generative AI 
	
	
\end{itemize}

\end{document}

