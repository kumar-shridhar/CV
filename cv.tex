% resume.tex
% vim:set ft=tex spell:

\documentclass[10pt,letterpaper]{article}
\usepackage[letterpaper,margin=0.75in]{geometry}
\usepackage[utf8]{inputenc}
\usepackage{mdwlist}
\usepackage[T1]{fontenc}
\usepackage{textcomp}
\usepackage{fontawesome}
\usepackage{tgpagella}
\usepackage{latexsym}
\usepackage{amssymb}
\usepackage{hyperref}
\usepackage[usenames, dvipsnames]{color}
\pagestyle{empty}
\setlength{\tabcolsep}{0em}

% indentsection style, used for sections that aren't already in lists
% that need indentation to the level of all text in the document
\newenvironment{indentsection}[1]%
{\begin{list}{}%
	{\setlength{\leftmargin}{#1}}%
	\item[]%
}
{\end{list}}

% opposite of above; bump a section back toward the left margin
\newenvironment{unindentsection}[1]%
{\begin{list}{}%
	{\setlength{\leftmargin}{-0.5#1}}%
	\item[]%
}
{\end{list}}

% format two pieces of text, one left aligned and one right aligned
\newcommand{\headerrow}[2]
{\begin{tabular*}{\linewidth}{l@{\extracolsep{\fill}}r}
	#1 &
	#2 \\
\end{tabular*}}

% make "C++" look pretty when used in text by touching up the plus signs
\newcommand{\CPP}
{C\nolinebreak[4]\hspace{-.05em}\raisebox{.22ex}{\footnotesize\bf ++}}

% and the actual content starts here
\begin{document}

\begin{center}
{\LARGE \color{MidnightBlue} \textbf{Kumar Shridhar}}

\url{ kumar-shridhar.github.io} \textbullet
\ \ \texttt{kmr.shridhr@gmail.com} \textbullet
\ \ +49 176 77659867
\\
Kurt-Schumacher Strasse 16, Kaiserslautern, Germany  67663 \\
\faGithub: @kumar-shridhar | \faLinkedin: @kumar-shridhar | \faMedium: @shridhar743 | \faTwitter: @jupyterai | \faGraduationCap: Kumar Shridhar
\end{center}


\hrule
\vspace{-0.4em}
\subsection*{\color{MidnightBlue}Publications}

\begin{enumerate}
	\parskip=0.1em
	
	\item Pedro Alonso*, \textbf{Kumar Shridhar*}, Denis Kleyko* ,  Evgeny Osipov, Marcus Liwicki (2019). \href {https://openreview.net/pdf?id=SJgzXaNFwS} {HyperEmbed: Trade-off Between Resources and Performance in NLP Tasks with Hyperdimensional Computing-based embedding of $n$-gram statistics}. \emph{Submitted to 8th Conference on International Conference on Learning Representations (ICLR 2020).}
	
	\item\textbf{Kumar Shridhar}, Felix Laumann, Marcus Liwicki (2019). \href{https://arxiv.org/abs/1806.05978}{Uncertainty Estimations by Softplus normalization in Bayesian Convolutional Neural Networks with Variational Inference}\footnote{\url{https://arxiv.org/abs/1806.05978}}. \emph{Submitted to 33rd Conference on Neural Information Processing Systems (NeurIPS 2019) Workshop on Bayesian Deep Learning.}
	
	\item Joonho Lee*, \textbf{Kumar Shridhar*} \footnote{* Equal Contribution}, Hideaki Hayashi, Brian Kenji Iwana, Seokjun Kang, Seiichi Uchida (2019). \href{https://arxiv.org/abs/1905.10761}{ProbAct: A Probabilistic Activation Function for Deep Neural Networks}\footnote{\url{https://arxiv.org/abs/1806.05978}}. \emph{ArXiv Preprint arXiv:1905.10761}
	
	\item\textbf{Kumar Shridhar}, Amit Sahu, Ayushman Dash, Pedro Alonso, Gustav Pihlgren, Vinay Pondeknath,  Gyorgy Kovacs, Fotini Simistira, Marcus Liwicki (2018). \href{https://arxiv.org/abs/1810.07150}{Subword Semantic Hashing for Intent Classification on Small Datasets}\footnote{\url{https://arxiv.org/abs/1810.07150}}. \emph{In Proceedings of IJCNN 2019, Budapest, Hungary.}
	
	\item Gyorgy Kovacs, Vanda Balogh, Purvanshi Mehta, \textbf{Kumar Shridhar}, Pedro Alonso, Marcus Liwicki (2019). \href{http://ceur-ws.org/Vol-2380/paper_244.pdf}{Author Profiling Using Semantic and Syntactic Features}\footnote{\url{http://ceur-ws.org/Vol-2380/paper_244.pdf}}. \emph{Conference and Labs of the Evaluation Forum (CLEF 2019), Lugano, Switzerland}
	
	%add abstract
\end{enumerate}

\hrule
\vspace{-0.4em}

\subsection*{\color{MidnightBlue}Masters Thesis}
\begin{itemize}
	\parskip=0.1em
	
	\item 
	\textbf{Kumar Shridhar},  Felix Laumann, Marcus Liwicki (2019). \href{https://arxiv.org/abs/1901.02731}{A Comprehensive guide to Bayesian Convolutional Neural Network with Variational Inference}\footnote{\url{https://arxiv.org/abs/1806.05978}}. \emph{ArXiv Preprint, arXiv:1901.02731.}
	
\end{itemize}

\hrule
\vspace{-0.4em}

\subsection*{\color{MidnightBlue}Education}

\begin{itemize}
	\parskip=0.1em
	
	\item 
	\headerrow
		{\textbf{\color{BrickRed}University of Kaiserslautern}}
		{\textbf{Kaiserslautern, Germany}}
	%add courses taken
	\\
	\headerrow
		{\emph{\color{OliveGreen}Department of Computer Science, Masters}}
		{\emph{04/2016 -- Present}}
	\begin{itemize*}
		\item Curriculum \footnote{\url{ https://www.cs.uni-kl.de/en/studium/studiengaenge/bm-inf/sp.ma/}} subjects: Machine Learning I, Machine Learning II, Very Deep Learning, Applications of Artificial Intelligence, Social Web Mining, 2D Computer Vision, Collaborative Intelligence, Embedded Intelligence, Document and Content Analysis, Linguistics and Language Processing, Neural Basis of Brain, Seminar, Project, Masters Thesis. \item Teaching assistant for {\emph{Very Deep Learning}} \footnote{\url{https://www.informatik.uni-kl.de/en/studium/lehrveranstaltungen/modulhb/#mod-89-7157}} coursework at TU Kaiserslautern under Prof. Marcus Liwicki.
	\end{itemize*}
	
	\item 
	\headerrow
		{\textbf{\color{BrickRed}Lule\aa \ University of Technology, Sweden}}
		{\textbf{Lule\aa \ , Sweden}}
	%add courses taken
	\\
	\headerrow
		{\emph{\color{OliveGreen}Student Researcher}}
		{\emph{02/2019 -- 03/2019}}
	\begin{itemize*}
	
		\item Worked with \href{https://www.ltu.se/staff/m/marliw-1.177225?l=en} {Prof. Marcus Liwicki} and \href{https://www.ltu.se/research/subjects/Maskininlarning?l=en}{EISLAB Machine Learning}, Luleå in NLP domain. We proposed \href{https://chatbotslife.com/know-your-intent-sota-results-in-intent-classification-8e1ca47f364c}{Subword Semantic Hashing technique for Intent classification}, and achieved state-of-the-art results on three standard text datasets (Chatbot, Web-Applications and Ubuntu datasets). 

        \item We also proposed a hyperdimensional computing based embeddings that achieves state of the art results in text classification while reducing the time and memory complexity by 10 to 100 folds. The work is under review at ICLR 2020.
	\end{itemize*}
	\newpage
	\item 
	\headerrow
		{\textbf{\color{BrickRed}Kyushu University}}
		{\textbf{Fukuoka, Japan}}
	%add courses taken
	\\
	\headerrow
		{\emph{\color{OliveGreen}International Researcher}}
		{\emph{04/2019 -- 05/2019}}
	\begin{itemize*}
		\item Worked with \href{http://human.ait.kyushu-u.ac.jp/~uchida/index-e.html} {Prof. Seiichi Uchida} in \href{http://human.ait.kyushu-u.ac.jp/index.html}{Human Interface Lab}, Fukuoka Japan on probabilistic activation function.  

        \item The work is under review at the moment and is further extended in the domain of Bayesian Neural Networks.
	\end{itemize*}
	
		\item 
	\headerrow
		{\textbf{\color{BrickRed}Fast.ai}}
		{\textbf{International Fellowship Student}}
	\\
	%reduce the point to one
	\headerrow
		{\emph{\color{OliveGreen}Deep Learning}}
		{\emph{2017 -- 2017}}
	\begin{itemize*}
		\item I learned to apply cutting-edge Deep Learning methods for Natural Language Processing, Computer Vision and Recommendation Systems to achieve state of the art results more efficiently. 

        \item The course helped a lot in understanding and experimenting with more deeply connected architectures with less computational power and to understand the underline thought behind to further improve it. 
        The primary library used was PyTorch which provides great flexibility in experimenting with new things.
	\end{itemize*}
	
	
\end{itemize}


\hrule
\vspace{-0.4em}

\subsection*{\color{MidnightBlue}Experience}

\begin{itemize}
	\parskip=0.1em
	
	
	\item
	\headerrow
		{\textbf {\href{https://www.botsupply.ai/}{\color{BrickRed}BOTSUPPLY}}}
		{\textbf{Copenhagen, Denmark}}
	\\
	\headerrow
		{\emph{\color{OliveGreen}Chief Research Scientist}}
		{\emph{10/2018 -- 12/2019}}
	\begin{itemize*}
		\item Developed a Natural Language Processing Framework \footnote{\url{https://www.botsupply.ai/natural-language-processing}} from scratch in 40+ languages   that powers all the customers chatbots at BotSupply\footnote{\url{https://www.botsupply.ai/}}. The framework supported Intent classification, Entity Recognition, Sentiment Analysis, and Language Translation that are on par with the state-of-the-art models.  
		\item My research work focused on solving the problem of catastrophic forgetting in a Neural Network.
	\end{itemize*}

	\item
	\headerrow
		{\textbf{\href{https://www.insiders-technologies.de/home.html}{\color{BrickRed}INSIDERS TECHNOLOGIES}}}
		{\textbf{Kaiserslautern, Germany}}
	\\
	\headerrow
		{\emph{\color{OliveGreen}Research Assistant}}
		{\emph{01/2018 -- 09/2018}}
	\begin{itemize*}	
		\item Worked with the Ovation Machine Learning Team for Conversational Intelligent Bots. My work involved understanding the client problem, design suitable solutions and architectures for them and to research on improving a model performance on scarce datasets. 
		\item Contributed to open source Ovation Framework for Conversational Intelligence \footnote{\url{https://github.com/mindgarage/Ovation}} in collaboration with Mindgarage and participated in Ovation Summer Academy 2017.
	\end{itemize*}

	\item
	\headerrow
		{\textbf{\href{http://mindgarage.ai/}{\color{BrickRed}MINDGARAGE}}}
		{\textbf{Kaiserslautern, Germany}}
	\\
	\headerrow
		{\emph{\color{OliveGreen}Researcher}}
		{\emph{2016 -- Present}}
	\begin{itemize*}
		\item My research focused on Bayesian Neural Networks, Continual Learning and Natural Language Processing.
		\item Led the organizational team at MindGarage: Assisted in students' projects and masters thesis, organized hackathons and research colloquiums, and maintained github and website.
	\end{itemize*}
	\item
	\headerrow
		{\textbf{\href{https://www.whizleads.com/}{\color{BrickRed}WHIZLEADS}}}
		{\textbf{Sydney, Australia}}
	\\
	\headerrow
		{\emph{\color{OliveGreen}Machine Learning Engineer}}
		{\emph{10/2016 -- 12/2016}}
	\begin{itemize*}
		\item Worked in development of a suite of sales solutions: insights about clients, lead generation, task and invoice management, and social media integration all integrated in an app that world real time.
		\item Used machine learning algorithms to generate up to date and meaningful insights about clients' personalities, mood, consumer needs, language style and values using social media data.
	\end{itemize*}

\end{itemize}

\hrule
\vspace{-0.4em}
\subsection*{\color{MidnightBlue} Research Collaborations}

\begin{itemize}
	\parskip=0.1em
	
	\item 
	\headerrow
		{\textbf{\href{http://www.mobile-industrial-robots.com/en/}{\color{BrickRed}Mobile Industrial Robots}}}
			{\textbf{}}
	\\
	\headerrow
		{\emph{\color{OliveGreen}Improvement of Object detection and localization systems in Mobile Industrial Robots}}
		{\emph{}}
	\begin{itemize*}
		\item Worked in the area of real-time Object detection in Mobile Industrial Robots using Nvidia Jetson devices and Raspberry Pi v2 cameras. Further, experimentation with Intel Movidius devices to reduce overall cost without a reduction in overall performance and accuracy. 

	\end{itemize*}
	
    	\item 
	\headerrow
		{\textbf{\href{https://www.jatana.ai/}{\color{BrickRed}Jatana AI}}}
			{\textbf{}}
	\\
	\headerrow
		{\emph{\color{OliveGreen}Research on learning from feedbacks in a coversational intelligent system}}
		{\emph{}}
	\begin{itemize*}
		\item Working together with researchers at Jatana to make the model learn from customer feedbacks automatically in order to improve the confidence of the low confidence queries replies.  

	\end{itemize*}
	
\end{itemize}

\hrule
\vspace{-0.4em}
\subsection*{\color{MidnightBlue}Other Notable Information}

\begin{itemize}
	\parskip=0.1em
	
	\item 
	\headerrow
		{{\textbf{ICLR}} 2020 Reviewer}
		{\emph{10/2019 -- Present}}
	\item 
	\headerrow
		{{\textbf{\href{https://www.kaggle.com/shridhar743}{Kaggle}}} Rank 5 -- Plant Seedling Identification}
		{\emph{11/2017 -- Present}}
	\item 
	\headerrow
		{{\textbf{\href{https://medium.com/@shridhar743}{Medium}}} Top Writer -- Artificial Intelligence (Over 200K reads)}
		{\emph{07/2017 -- 09/2017}}
	\item 
	\headerrow
		{Member of Botsupply IBM Award Winner 2017 Team}
		{\emph{11/2017}}
	\item 
	\headerrow
		{\textbf{GitHub} over 800 stars and over 200 forks}
		
	\item
	\headerrow
	    {\textbf{Open Source Contributions}:
	Facebook Duckling, ContinualAI, FastAI}
		

\end{itemize}

\hrule
\vspace{-0.4em}
\subsection*{\color{MidnightBlue}Languages and Technologies}

\begin{indentsection}{\parindent}
\hyphenpenalty=1000
\begin{description*}
	\item[\color{BrickRed}Programming Languages:]
	Python, C, C++
	\item[\color{BrickRed}Technologies:]
	PyTorch, Keras, TensorFlow, Pyro, SciPy, NumPy, scikit-learn, FastText, NLTK, RASA, SpaCy, UNIX, Docker, Git, \LaTeX, Jupyter Notebook
	\item[\color{BrickRed}Natural Languages:]
	Native in English and Hindi, intermediate in German	
\end{description*}
\end{indentsection}

\hrule
\vspace{-0.4em}
\subsection*{\color{MidnightBlue}Talks and Presentations}

\begin{itemize}
	\parskip=0.1em
	
	\item {\color{BrickRed} IJCNN 2019}, July 2019 : Oral presentation of '\textit{SubWord Semantic Hashing for Intent Classification}' Paper.
	
	\item {\color{BrickRed} Copenhagen Chatbots and AI Meetup}, June 2017 - Present : Best practices, ongoing research in NLP and combination of chatbots with design process to achieve best results.
	
	\item {\color{BrickRed} Lule\aa\ Technical University}, Lule\aa\ Sweden, August 2018: Know your Intent: Intent classification using Semantic Hashing
	
	\item {\color{BrickRed} iMuSciCA}, Athens Greece, May 2018: Generative Adversarial Networks for Semantic Segmentation 
	\item {\color{BrickRed} Technical University Kaiserslautern}, March 2018: Empirical Evaluation of DenseNet 
		
	\item {\color{BrickRed} Ovation Summer Academy}, September 2017: NER using synthetic datasets 
	
	\item {\color{BrickRed} TechFestival}, Copenhagen Denmark, September 2017: Generative AI 
	
\end{itemize}

\hrule
\vspace{-0.4em}
\subsection*{\color{MidnightBlue}References}

\begin{itemize}
	\parskip=0.1em

    \item \href{https://www.ltu.se/staff/m/marliw-1.177225?l=en}{\color{BrickRed} Prof. Marcus Liwicki}, Professor and Head of Subject, Chaired Professor, Luleå University of Technology, Sweden. 
    
    \item \href{http://human.ait.kyushu-u.ac.jp/~uchida/index-e.html}{\color{BrickRed} Prof. Seiichi Uchida}, Distinguished Professor, Department of Advanced Information Technology, Kyushu University, Japan. 
    
    \item \href{https://www.imperial.ac.uk/people/f.laumann18}{\color{BrickRed} Felix Laumann}, Research Postgraduate, Imperial College, London, UK.
    
    \item \href{https://www.linkedin.com/in/assersmidt/}{\color{BrickRed} Asser Smidt}, Founder and CEO, BotSupply, Copenhagen, Denmark.
    

\end{itemize}

\end{document}

